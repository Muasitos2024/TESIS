\newpage
\section*{Abstract}

Efficient management of water resources is aided by the use of approaches such as geostatistics, which allows a deep understanding of the spatial variability of physicochemical parameters in water bodies. This study focused on estimating physicochemical parameters to assess water quality in the Pacucha lagoon, located in the Apurimac Region, Peru, using geostatistical methods. Critical parameters such as temperature, pH and dissolved oxygen were studied from samples taken at 35 points distributed in the lagoon. The methodology included an exploratory analysis to identify descriptive statistics and correlations, followed by a variogram analysis in which theoretical models were fitted to the semivariograms: the spherical model was used for temperature and dissolved oxygen, while the Gaussian model was used for pH. In the final phase, the kriging technique was applied to estimate parameters at unsampled locations, revealing spatial patterns of variability and areas of concern near human activities. The results highlight how interactions between variables such as dissolved oxygen and temperature, affected by anthropogenic factors, influence water quality. This study highlights the effectiveness of geostatistics in gaining a comprehensive understanding of water quality, underlining the importance of data-driven management approaches for water resource conservation.

\textbf{Keywords:} Geostatistics, Kriging, Water quality, Pacucha lagoon, Variographic analysis, Physicochemical parameters.
