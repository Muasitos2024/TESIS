\newpage
\section*{Resumen}

La administración eficiente de los recursos hídricos se ve favorecida por el uso de enfoques como la geoestadística, que permite una comprensión profunda de la variabilidad espacial de los parámetros fisicoquímicos en cuerpos de agua. Este estudio se centró en estimar los parámetros fisicoquímicos para evaluar la calidad del agua en la laguna de Pacucha, situada en la Región de Apurímac, Perú, utilizando métodos geoestadísticos. Se estudiaron parámetros críticos como la temperatura, el pH y el oxígeno disuelto, a partir de muestras tomadas en 35 puntos distribuidos en la laguna. La metodología incluyó un análisis exploratorio para identificar estadísticas descriptivas y correlaciones, seguido de un análisis variográfico en el que se ajustaron modelos teóricos a los semivariogramas: el modelo esférico se empleó para la temperatura y el oxígeno disuelto, mientras que el modelo gaussiano se utilizó para el pH. En la fase final, se aplicó la técnica de kriging para estimar parámetros en ubicaciones no muestreadas, revelando patrones espaciales de variabilidad y áreas de preocupación cerca de actividades humanas. Los resultados resaltan cómo las interacciones entre variables como el oxígeno disuelto y la temperatura, afectadas por factores antropogénicos, influyen en la calidad del agua. Este estudio destaca la eficacia de la geoestadística para obtener una comprensión integral de la calidad del agua, subrayando la importancia de enfoques de gestión basados en datos para la conservación de los recursos hídricos.


\textbf{Palabras clave:} Geoestadística, Kriging, Calidad del agua, Laguna de Pacucha, Análisis variográfico, Parámetros fisicoquímicos.



