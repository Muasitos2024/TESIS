\newpage
\section*{Introduccion}
 
La Laguna de Pacucha, situada en la región meridional de Perú, es un recurso hídrico muy apreciado que exhibe la belleza natural y la biodiversidad de la zona. Más allá de su evidente esplendor, la laguna desempeña un papel crucial en la sostenibilidad medioambiental, la economía local y la cohesión social de las comunidades circundantes. Sin embargo, ninguna masa de agua dulce es inmune a los retos a los que se enfrentan muchas masas de agua en el mundo actual. Actividades humanas como la agricultura, el turismo y el uso del suelo, así como los efectos del cambio climático, imponen una carga cada vez mayor sobre las aguas de la laguna, poniendo en peligro la calidad y estabilidad del ecosistema acuático.

En este contexto, es imperativo desarrollar estrategias de gestión basadas en un conocimiento profundo y científico de las condiciones de la laguna. Esta tesis se alinea con este requisito proponiendo una evaluación detallada de la calidad del agua mediante la estimación de parámetros fisicoquímicos utilizando técnicas geoestadísticas avanzadas. Parámetros fisicoquímicos como la temperatura, el pH y el oxígeno disuelto son indicadores críticos de la salud ecológica de la laguna y proporcionan información esencial para la toma de decisiones en la gestión de los recursos hídricos.

La hipótesis general de la investigación sugiere que la aplicación de técnicas geoestadísticas proporcionará estimaciones precisas de los parámetros fisicoquímicos, permitiendo así una evaluación integral de la calidad del agua en la laguna de Pacucha. Las hipótesis específicas indican que la distribución de estos parámetros varía entre los diferentes puntos de muestreo y que las técnicas geoestadísticas permitirán identificar variaciones significativas en las zonas de la laguna, proporcionando una base sólida para futuras intervenciones de manejo y conservación.

La metodología implementada emplea un enfoque geoestadístico para el muestreo y el análisis de datos. Se recogerán muestras en 35 puntos distintos dentro de la laguna de Pacucha. El análisis comenzará con un estudio exploratorio para identificar estadísticas descriptivas y correlaciones. Posteriormente, se realizará un análisis variográfico donde se ajustarán modelos teóricos a los semivariogramas: el modelo esférico se aplicará para la temperatura y el oxígeno disuelto, mientras que el modelo gaussiano se utilizará para el pH. La fase final utilizará la técnica de kriging para estimar parámetros en ubicaciones no muestreadas, revelando patrones espaciales de variabilidad y áreas de preocupación cercanas a actividades humanas.

Este estudio no solo promete revelar la estructura espacial de los parámetros fisicoquímicos, sino que también establece un precedente para la aplicación de métodos analíticos en la gestión ambiental de los cuerpos de agua. La tesis representa un esfuerzo significativo para integrar técnicas analíticas avanzadas en la evaluación y gestión ambiental de cuerpos de agua vitales, como la Laguna de Pacucha. Los resultados no solo tendrán implicaciones locales, sino que también contribuirán a las prácticas globales de gestión del agua, sumándose al cuerpo de conocimientos en ciencias ambientales y sirviendo como modelo para la protección y gestión sostenible de ecosistemas acuáticos similares.