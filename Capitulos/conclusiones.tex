\chapter{CONCLUSIONES Y RECOMENDACIONES}

\begin{comment}
    

\section{Conclusiones}

\begin{enumerate}
    \item \textbf{Eficacia de la Geoestadística:}\\
    El estudio no solo resalta la eficacia de las técnicas geoestadísticas, sino que también destaca su versatilidad y adaptabilidad en contextos ambientales complejos. Kriging, por ejemplo, proporciona no solo estimaciones confiables en ubicaciones no muestradas, sino que también muestra cómo las variables ambientales interactúan espacialmente al incorporar modelos de variogramas. Esta técnica permite una comprensión más profunda de las dinámicas espaciales y se puede adaptar para incorporar múltiples fuentes de datos, lo que mejora la precisión y utilidad de las estimaciones. La inclusión de análisis de sensibilidad y validación cruzada fortalece la confianza en las predicciones y ofrece un marco sólido para tomar decisiones basadas en datos.
    
    \item \textbf{Variabilidad Espacial y Calidad del Agua:}\\
    La variabilidad espacial en la calidad del agua no es solo un fenómeno observado; refleja una compleja red de interacciones entre factores ecológicos, geológicos y antropogénicos. Identificar áreas con alteraciones significativas en parámetros críticos no solo ayuda con la planificación de la gestión ambiental, sino que también prioriza intervenciones. Profundizar en modelos espaciales que consideren factores, como el uso del suelo, la proximidad a fuentes de contaminación y tendencias históricas, puede revelar patrones aún más complejos. Además, este estudio podría expandirse para explorar las implicaciones a largo plazo de estas variaciones, como sus posibles efectos en la biodiversidad acuática y la salud humana.
    
    \item \textbf{Interrelaciones entre Parámetros:}\\
    Las correlaciones entre parámetros, como el oxígeno disuelto y la temperatura, son solo la punta del iceberg. Un análisis detallado de estas interacciones puede revelar los mecanismos subyacentes que afectan la calidad del agua. Por ejemplo, explorar modelos dinámicos que reflejen cómo los cambios estacionales y diarios afectan estas interacciones podría ofrecer nuevas perspectivas sobre la resiliencia de los ecosistemas acuáticos. Integrar estudios sobre las respuestas biológicas a estos cambios químicos y físicos puede proporcionar una perspectiva más holística. Comprender estas interacciones es crucial para diseñar estrategias de monitoreo y gestión adaptativas que puedan predecir y mitigar impactos negativos.
\end{enumerate}

\section{Recomendaciones}

\begin{enumerate}
    \item \textbf{Monitoreo Continuo:}\\
    Implementar un programa de monitoreo continuo es esencial para entender y gestionar la dinámica de la calidad del agua a lo largo del tiempo. Las técnicas geoestadísticas avanzadas no solo permiten detectar cambios, sino también predecir tendencias y posibles eventos críticos. Este monitoreo debe incorporar tanto sensores in situ como técnicas de teledetección para una cobertura amplia y en tiempo real. La instalación de estaciones automáticas de muestreo y análisis en tiempo real podría proporcionar alertas tempranas de deterioro en la calidad del agua. Además, establecer un modelo predictivo basado en datos históricos y actuales podría ayudar a anticipar eventos adversos, permitiendo una respuesta más rápida y efectiva.
    
    \item \textbf{Gestión Basada en Datos:}\\
    Una gestión efectiva del agua requiere no solo información actual, sino también una comprensión histórica y una capacidad de predecir el futuro. Incorporar los resultados de su estudio en modelos de gestión y políticas puede mejorar significativamente la eficacia de las intervenciones. Esto incluye la identificación de áreas críticas para la restauración o protección y la optimización del uso y asignación de recursos hídricos. Además, los datos pueden ayudar a desarrollar estándares y regulaciones más precisos y personalizados para la conservación del agua. La colaboración con instituciones científicas y tecnológicas para el desarrollo de herramientas y plataformas de análisis de datos puede facilitar una gestión basada en datos.
    
    \item \textbf{Concienciación y Participación Comunitaria:}\\
    Para lograr un éxito a largo plazo en cualquier programa de gestión del agua, la concienciación y participación comunitaria son cruciales. Implementar programas educativos que demuestren el impacto de las actividades humanas en la calidad del agua y la importancia de su conservación puede promover prácticas más sostenibles. Establecer plataformas de colaboración entre comunidades, autoridades locales y expertos puede mejorar la gobernanza del agua. Además, involucrar a la comunidad en la vigilancia ciudadana, donde los locales pueden contribuir a la recopilación de datos y la vigilancia, no solo aumenta la disponibilidad de datos, sino que también fomenta un sentido de propiedad y responsabilidad hacia los recursos hídricos locales.
\end{enumerate}

\end{comment}
\section{Conclusiones}
\begin{enumerate}
    \item En este estudio se aplicaron exitosamente técnicas geoestadísticas para estimar los parámetros fisicoquímicos de temperatura, pH y oxígeno disuelto en la Laguna de Pacucha. Los resultados obtenidos indican que ninguno de los parámetros evaluados supera las normas de calidad ambiental establecidas. Sin embargo, se encontró que los valores de oxígeno disuelto están cerca del límite mínimo permitido, lo que podría indicar una vulnerabilidad a largo plazo en la calidad del agua, especialmente bajo influencias antropogénicas y ambientales. La estimación Kriging permitió identificar las distribuciones espaciales de estos parámetros, proporcionando una evaluación completa y detallada de la calidad del agua. Estos resultados subrayan la importancia de vigilar continuamente estos parámetros y de aplicar estrategias proactivas de gestión ambiental para garantizar la conservación y sostenibilidad de este valioso recurso hídrico.
    \item Se llevó a cabo con éxito la recogida de datos sobre parámetros físicos y químicos de la calidad del agua en diferentes puntos de muestreo distribuidos a lo largo de la laguna de Pacucha. El muestreo sistemático y metódico, basado en el Plan de Monitoreo de Recursos Hídricos, garantizó la exactitud y pertinencia de los datos obtenidos. La implementación de prácticas de control de calidad, incluyendo la calibración de instrumentos, aseguró la confiabilidad y consistencia de los resultados. La recogida de 35 muestras proporcionó una representatividad adecuada para el análisis geoestadístico. Los resultados obtenidos, comparables con los de estudios anteriores realizados en distintos lagos, ponen de relieve la universalidad y eficacia de los métodos utilizados, al tiempo que subrayan la necesidad de adaptar las técnicas a las características locales específicas. Este enfoque permitió la identificación precisa de zonas potencialmente vulnerables, facilitando el desarrollo de estrategias de gestión adaptadas para la conservación del recurso hídrico en la Laguna de Pacucha.
    \item Se aplicó un análisis geoestadístico a los datos recolectados, permitiendo la estimación de parámetros fisicoquímicos en diferentes áreas de la Laguna de Pacucha. Utilizando el método de kriging, se identificaron diferencias espaciales en los niveles de temperatura, oxígeno disuelto y pH. Este análisis permitió detectar áreas con riesgo ecológico debido a influencias antropogénicas, lo cual es consistente con hallazgos de estudios similares en otras regiones. Se destacó la eficacia y universalidad de las técnicas geoestadísticas para proporcionar estimaciones precisas en zonas no vigiladas. Sin embargo, fue evidente la necesidad de ajustes metodológicos para adaptarse a las especificidades locales de la Laguna de Pacucha, enfatizando la importancia de un monitoreo continuo y adaptativo. Los resultados obtenidos refuerzan el valor del kriging como herramienta crucial para la planificación y gestión ambiental, proporcionando una base sólida para la conservación y gestión eficaz de los recursos hídricos de la Laguna de Pacucha.
    \item Se realizó una evaluación de la calidad del agua de la laguna de Pacucha, comparando los resultados obtenidos con las normas y reglamentos ambientales vigentes. La aplicación de técnicas de interpolación geoestadística permitió estimar con precisión parámetros físico-químicos esenciales, como la temperatura, el pH y el oxígeno disuelto. Los resultados mostraron que, aunque la mayoría de las áreas de la laguna cumplen los Estándares de Calidad Ambiental (ECA), algunas zonas presentan temperaturas y niveles de oxígeno disuelto cercanos al límite mínimo permitido, lo que las hace vulnerables a la contaminación, especialmente en áreas con influencias antropogénicas. Estos resultados concuerdan con estudios previos realizados en otras masas de agua, lo que subraya la importancia de la monitorización continua y la gestión adaptativa para prevenir la degradación de la calidad del agua. La metodología aplicada en esta investigación proporciona una base sólida para futuras intervenciones de remediación y políticas de conservación, asegurando que la calidad del agua en la Laguna de Pacucha permanezca dentro de estándares aceptables y responda adecuadamente a las condiciones cambiantes y a las influencias humanas.
\end{enumerate}


\section{Recomendaciones}

\begin{enumerate}
    \item En cuanto a la estimación de parámetros físico-químicos para evaluar la calidad del agua de la Laguna de Pacucha utilizando técnicas geoestadísticas, se recomienda implementar un programa de monitoreo continuo que utilice estas técnicas avanzadas de análisis espacial. Este programa debe incluir la calibración periódica de los instrumentos y la toma sistemática de datos en puntos de muestreo bien distribuidos. Además, es esencial realizar análisis temporales para detectar posibles tendencias de deterioro de la calidad del agua, sobre todo en el caso de parámetros como el oxígeno disuelto, que se aproximan a las normas mínimas de calidad ambiental. La integración de estas prácticas permitirá una gestión más efectiva y proactiva de los recursos hídricos, facilitando la implementación de medidas de conservación y remediación para asegurar la sostenibilidad ambiental de la Laguna de Pacucha.
    \item Para recoger los parámetros físico-químicos de la calidad del agua en los diferentes puntos de muestreo distribuidos en la laguna de Pacucha, se recomienda establecer un protocolo de muestreo sistemático y riguroso. Este protocolo debe incluir un plan detallado de los puntos de muestreo para garantizar una cobertura exhaustiva de toda la laguna. Además, es esencial la calibración periódica de los instrumentos de medición y el control de calidad mediante muestras de referencia. La formación del personal responsable de la recogida de datos también es fundamental para garantizar la coherencia y precisión de las mediciones.
    \item Para aplicar el análisis geoestadístico a los datos recolectados y estimar los parámetros físico-químicos en diferentes áreas de la laguna de Pacucha, se recomienda utilizar técnicas avanzadas de análisis espacial como kriging. Es crucial validar los modelos geoestadísticos mediante técnicas de validación cruzada para asegurar su precisión y confiabilidad. Además, debe considerarse la inclusión de variables adicionales que puedan influir en la calidad del agua, como las actividades antropogénicas y los factores climáticos, para mejorar la precisión de las estimaciones y obtener una comprensión más completa de la dinámica del ecosistema acuático.
    \item Para evaluar la calidad del agua de la Laguna de Pacucha comparando los resultados obtenidos con las normas y regulaciones ambientales vigentes, se recomienda implementar un sistema de monitoreo
    \end{enumerate}