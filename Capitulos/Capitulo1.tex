\chapter{PLANTEAMIENTO DEL ESTUDIO}  

\section{Planteamiento y Formulación del Problema}

\subsection{Planteamiento del Problema}

En la era actual, caracterizada por un rápido crecimiento poblacional y cambios en los estilos de vida, la calidad y accesibilidad del agua enfrentan desafíos significativos a nivel mundial. Este fenómeno se hace evidente en la creciente demanda de recursos hídricos, particularmente en áreas como la Laguna Pacucha. La expansión de la población y las modificaciones en los patrones de consumo han incrementado el uso del agua en zonas tanto urbanas como rurales. Este problema se intensifica debido a que la infraestructura hídrica existente no es suficiente para cubrir las crecientes necesidades, resultando en una escasez de este recurso vital. Además, el aumento en el consumo de agua ha llevado a una mayor descarga de aguas residuales contaminadas, lo que hace necesario el desarrollo de métodos de tratamiento avanzados para su remediación y reutilización.

La degradación de la calidad del agua tiene repercusiones significativas en los ámbitos económico y social. La Laguna de Pacucha, un ecosistema acuático crucial en la región de Apurímac, Perú, ejemplifica claramente esta situación, enfrentando actualmente amenazas tanto naturales como antropogénicas. La sobreexplotación del agua y actividades como la urbanización y la agricultura intensiva han provocado un deterioro en la calidad del agua en esta región. Específicamente, la Laguna de Pacucha está en riesgo de contaminación debido a actividades humanas como el turismo, la ganadería y la agricultura, lo que compromete la calidad de este recurso esencial.

En este contexto, el objetivo de esta investigación es abordar el desafío de evaluar la calidad del agua en la Laguna Pacucha mediante el uso de técnicas geoestadísticas. Estas técnicas permitirán analizar y predecir la variabilidad espacial de parámetros fisicoquímicos como el oxígeno disuelto, la temperatura y el pH, los cuales son importantes para determinar la calidad del agua. El empleo de la geoestadística es fundamental para rastrear la dispersión espacial de estos parámetros e identificar áreas vulnerables. Esto es esencial para anticipar cambios en la calidad del agua e identificar posibles fuentes de contaminación, especialmente en áreas de la Laguna Pacucha que aún no han sido muestreadas.

Este estudio contribuirá de manera significativa a la comprensión y gestión de la calidad del agua en la Laguna Pacucha. Proporcionará una base sólida para la toma de decisiones informadas y la implementación de estrategias efectivas de mitigación. La evaluación continua de la calidad del agua, a través del uso de parámetros fisicoquímicos, es esencial para mitigar las fuentes de contaminación y asegurar la preservación de este recurso hídrico vital para la supervivencia de las comunidades locales y las futuras generaciones. En consecuencia, este proyecto adopta un enfoque sostenible y responsable en la gestión ambiental, orientado a la conservación y uso prudente de los recursos hídricos.

\subsection{Formulación del Problema}

\subsubsection{Problema General}
¿Cómo se pueden estimar los parámetros fisicoquímicos y evaluar la calidad del agua en la Laguna de Pacucha utilizando técnicas geoestadísticas?

\subsubsection{Problemas Específicos}
\begin{itemize}
    \item ¿Cómo se puede llevar a cabo la recolección de datos de los parámetros fisicoquímicos de la calidad del agua en diversos puntos de muestreo en la Laguna de Pacucha?
    \item ¿Cuál es la metodología más adecuada para llevar a cabo un análisis geoestadístico de los datos recopilados, con el fin de lograr una estimación precisa de los parámetros fisicoquímicos en distintas áreas de la Laguna Pacucha?
    \item ¿Cómo se puede llevar a cabo una evaluación de la calidad del agua en la Laguna de Pacucha, considerando la comparación de los resultados obtenidos a través de la estimación con los estándares y regulaciones ambientales que están vigentes en la actualidad?
\end{itemize}



\section{Objetivos}

\subsection{Objetivo general}

Estimar los parámetros fisicoquímicos para evaluar la calidad del agua en la laguna de Pacucha mediante la técnica geoestadística.

\subsection{Objetivos específicos}

\begin{itemize}
    \item Realizar la recolección de datos de los parámetros fisicoquímicos de la calidad del agua en distintos puntos de muestreo distribuidos en la Laguna de Pacucha.
    \item Aplicar un análisis geoestadístico a los datos recopilados, para realizar una estimación de los parámetros fisicoquímicos en diferentes áreas de la laguna Pacucha.
    \item Realizar una evaluación de la calidad del agua en la laguna de Pacucha, a través de la comparación de los resultados obtenidos mediante la estimación con los estándares y regulaciones ambientales actualmente vigentes.
\end{itemize}

\section{Justificación e Importancia}
\subsection{Justificación Metodológica}

Según Bernal \cite{bernal2010metodologia}, en la investigación científica, la justificación metodológica se establece cuando un proyecto introduce un método innovador o una estrategia novedosa para generar conocimiento que sea tanto válido como confiable.

En este estudio sobre la estimación de parámetros fisicoquímicos mediante técnicas geoestadísticas, se fundamenta en la aplicación rigurosa del método científico como enfoque general. Específicamente, se empleará un diseño metodológico que es observacional, retrospectivo, transversal y analítico para la realización del estudio.

\subsection{Justificación Ambiental}

Según \textcite{book:2636572} la justificación ambiental se fundamenta en la integración de la relevancia del enfoque sustentable, el cual abarca aspectos ambientales, económicos y sociales, y promueve la sostenibilidad de los recursos naturales.

En este contexto, la Laguna de Pacucha, un ecosistema frágil y valioso, está siendo afectada por diversos factores ambientales, incluyendo la contaminación del agua. Es crucial llevar a cabo una evaluación precisa de la calidad del agua para entender cómo la contaminación impacta el ecosistema y para implementar las medidas necesarias para su protección. La técnica geoestadística permite realizar estimaciones exactas de los parámetros fisicoquímicos de la calidad del agua en áreas no monitoreadas, ofreciendo una visión integral de la calidad del agua en toda la laguna. Además, la combinación de información geográfica y estadística facilita la identificación de las zonas con mayores problemas de calidad del agua, lo que permite priorizar las acciones necesarias para proteger el ecosistema. El uso de esta técnica es indispensable para evaluar la calidad del agua de la Laguna de Pacucha, asegurando así el uso sostenible del recurso y la protección del equilibrio ecológico del área.

\subsection{Justificación Socio económica}

Según Carrasco \cite{carrasco2005metodologia}, la justificación socioeconómica se basa en los beneficios y utilidades que los resultados de la investigación aportan a la población. Estos resultados constituyen una base fundamental y un punto de partida para desarrollar proyectos de mejora social y económica en la comunidad. En este contexto, evaluar con precisión la calidad del agua de la Laguna Pacucha es crucial para garantizar su uso sostenible y proteger la salud pública, lo que a su vez genera un impacto económico positivo.

La aplicación de técnicas geoestadísticas proporciona un método más preciso y eficiente para estimar los parámetros de calidad del agua en comparación con otras técnicas, lo que se traduce en un ahorro de costos en la evaluación de la calidad del agua. Además, al identificar las áreas con problemas de calidad del agua, se pueden implementar medidas más efectivas para proteger el recurso y prevenir daños ambientales y de salud pública que podrían ser costosos.




\subsection{Justificación Social}

La justificación social, según Ñaupas \cite{naupas2014metodologia}, se fundamenta en la premisa de que la investigación debe abordar problemas sociales que afectan a un grupo específico.

La evaluación de la calidad del agua en la Laguna de Pacucha tiene un impacto significativo en la vida de las personas que dependen de este recurso para sus necesidades cotidianas, tales como la agricultura, la industria y el consumo humano. La aplicación de técnicas geoestadísticas permite estimar con precisión los parámetros fisicoquímicos de la calidad del agua en áreas no muestreadas, proporcionando una visión más completa de la calidad del agua en toda la laguna. Además, al identificar las áreas con problemas de calidad del agua, se pueden implementar medidas para proteger el recurso y prevenir la exposición a sustancias tóxicas que podrían tener un impacto negativo en la salud humana. El uso de esta técnica en la evaluación de la calidad del agua de la Laguna de Pacucha es esencial para garantizar el uso sostenible del recurso y proteger la salud y el bienestar de las personas que dependen de él.

\section{Importancia}

En el contexto de crecientes desafíos ambientales y climáticos globales, la investigación centrada en la Laguna de Pacucha es crucial. El cambio climático y la gestión inadecuada de los recursos hídricos han aumentado la vulnerabilidad de ecosistemas acuáticos como la Laguna de Pacucha, amenazando la biodiversidad, la seguridad hídrica y la sostenibilidad de las comunidades locales. La estimación de parámetros fisicoquímicos mediante métodos geoestadísticos proporciona una herramienta valiosa para entender y predecir los impactos de estas amenazas en la calidad del agua. Este estudio no solo busca ampliar el conocimiento científico y apoyar la toma de decisiones informadas, sino también ser un catalizador para estrategias de gestión adaptativa y sostenible. Al evaluar la calidad del agua en la Laguna de Pacucha, esta investigación se alinea con los esfuerzos globales y regionales para mitigar los efectos del cambio climático, promover prácticas de gestión sostenible y asegurar el acceso a agua de calidad, abordando el Objetivo de Desarrollo Sostenible número seis y respondiendo a las urgencias locales y globales actuales.

El documento conmemorativo del Día Mundial del Agua de la UNESCO destaca el impacto negativo del cambio climático sobre la disponibilidad y pureza del agua, un tema estrechamente relacionado con esta tesis. La Laguna de Pacucha, como muchos otros ecosistemas, sufre los efectos adversos de eventos climáticos severos como tormentas, inundaciones y sequías, exacerbados por el cambio climático, lo cual afecta tanto la calidad como la cantidad de agua disponible \cite{UN2020Agua}.

Esta investigación reconoce que una gestión inadecuada del agua, especialmente frente a los crecientes desafíos climáticos, puede obstaculizar el logro del Objetivo de Desarrollo Sostenible número seis, que busca garantizar el acceso universal a agua potable segura y asequible para 2030. Utilizando mediciones precisas de parámetros fisicoquímicos y métodos geoestadísticos, esta tesis no solo caracteriza el estado actual de la calidad del agua en la Laguna de Pacucha, sino que también proporciona una base científica para mejorar la gestión del agua y la toma de decisiones relacionadas con los recursos hídricos.

Al centrarse en la calidad del agua en un entorno vulnerable a los impactos del cambio climático, este trabajo contribuye a una mejor comprensión de cómo estos fenómenos pueden alterar los ecosistemas acuáticos. Además, ofrece valiosas perspectivas para fundamentar políticas y estrategias dirigidas a una gestión sostenible del agua, en consonancia con los esfuerzos globales para enfrentar los retos ambientales y cumplir con los compromisos de sostenibilidad.

\begin{table}[H]
\centering
\caption{Relación entre los Objetivos de Desarrollo Sostenible (ODS) y la tesis sobre la calidad del agua en la Laguna de Pacucha}
\begin{tabular}{p{0.7cm}p{3cm}p{9cm}}
\hline
\textbf{ODS} & \textbf{Objetivo}                      & \textbf{Relación con la Tesis}                                                                                                           \\ \hline
6            & Agua limpia y saneamiento              & Mejora la calidad del agua y su gestión sostenible. La tesis evalúa parámetros fisicoquímicos para asegurar la calidad del agua en la laguna. \\ \hline
14           & Vida submarina                         & Conserva y usa sosteniblemente los recursos acuáticos. La tesis ayuda a preservar la vida acuática evaluando la calidad del agua.            \\ \hline
15           & Vida de ecosistemas terrestres         & Protege y promueve el uso sostenible de ecosistemas. La tesis examina el impacto de los parámetros fisicoquímicos en la biodiversidad.       \\ \hline
11           & Ciudades y comunidades sostenibles     & Crea comunidades resilientes y sostenibles. La tesis mejora la comprensión de cómo la calidad del agua impacta la sostenibilidad local.      \\ \hline
\end{tabular}
\label{table:ODS_relation}
\end{table}

\begin{comment}
    
\subsection{Cambio Climático y Distribución del Agua}
Se prevé que el cambio climático altere la distribución mundial del agua, con importantes variaciones en las predicciones del Grupo Intergubernamental de Expertos sobre el Cambio Climático (IPCC) incluso dentro de una misma región\cite{dwAguaEscasa}. En el contexto de la Laguna de Pacucha, esto podría significar cambios en los patrones de lluvia y evaporación, afectando directamente los parámetros fisicoquímicos que son cruciales en mi estudio para la evaluación de la calidad del agua.

\subsection{Necesidad de Acción Inmediata}
Dada la variabilidad de las proyecciones sobre el cambio climático y sus efectos directos sobre el ciclo hidrológico, es necesario actuar de inmediato\cite{dwAguaEscasa}. Mi tesis contribuye a esta acción al proporcionar un entendimiento  de la calidad actual del agua en la Laguna de Pacucha, ofreciendo una base para estrategias de adaptación y mitigación efectivas en la región una vez estimadas los parametros en estudio.
\end{comment}


section{Delimitación del estudio}

Este estudio se centra en la Laguna de Pacucha, situada en el distrito de Pacucha, provincia de Andahuaylas, región de Apurímac, Perú. La delimitación geográfica abarca la laguna y su entorno inmediato, incluyendo los centros poblados cercanos y la microcuenca asociada. Este enfoque asegura una comprensión detallada tanto de las características fisicoquímicas del agua como de las dinámicas sociales y demográficas de las comunidades circundantes.

El muestreo se realizará exclusivamente en la Laguna de Pacucha. Se evaluarán parámetros fisicoquímicos como el oxígeno disuelto, el pH y la temperatura, esenciales para determinar la calidad del agua. Estos parámetros serán medidos con precisión para reflejar las condiciones actuales de la laguna.

La delimitación temporal del estudio es de naturaleza transversal, con la recopilación de datos y el análisis planificados para el mes de noviembre. Este enfoque temporal proporciona una instantánea de las condiciones ambientales y fisicoquímicas en un período específico, permitiendo una evaluación concentrada y precisa de la calidad del agua.





\section{Hipótesis y variables}

\subsection{Hipótesis general}
La aplicación de la técnica geoestadística permitirá una estimación de los parámetros fisicoquímicos, lo que posibilitará la evaluación de la calidad del agua en la Laguna de Pacucha.
\subsection{Hipótesis específicas}
\begin{itemize}
    \item La distribución de los parámetros fisicoquímicos de la calidad del agua varía entre los diferentes puntos de muestreo en la Laguna de Pacucha.
    \item Mediante la aplicación de técnicas geoestadísticas, se pueden identificar variaciones en los parámetros fisicoquímicos en diferentes zonas de la Laguna de Pacucha.
    \item La evaluación de la calidad del agua en la Laguna de Pacucha, basada en la estimación de los parámetros fisicoquímicos, revelará áreas de mejora para cumplir con las normativas y regulaciones ambientales actuales.
\end{itemize}

\subsection{Variables}

En esta investigación, dado su carácter descriptivo y predictivo, se maneja una sola variable principal, que es la calidad del agua en la Laguna de Pacucha. La evaluación de la calidad del agua se realiza a través de la estimación de parámetros fisicoquímicos específicos.

\subsubsection{Variable principal}
\begin{itemize}
\item Calidad del agua: Evaluada mediante los parámetros fisicoquímicos de oxígeno disuelto, pH y temperatura.
\end{itemize}