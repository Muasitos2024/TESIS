\begin{titlepage}
% Ajustar márgenes solo para esta página
    \newgeometry{top=1in, bottom=0.1in, left=1in, right=1in}

    % Ajuste de la posición del logo en la parte superior izquierda
    \vspace*{-2cm} % Ajustar la distancia entre el borde superior y el logo
    \hspace*{-2cm} % Mover el contenido ligeramente hacia la izquierda
    \includegraphics[scale=0.6]{Portada/uc.pdf}\\[0.1cm] % Logo como membrete en la parte superior izquierda
    
    % Ajustar el espacio entre el logo y "FACULTAD DE INGENIERÍA"
    \vspace{0.1cm} % Reducir el espacio entre el logo y el título

    % Alinear el resto del contenido centrado
    \begin{center}
        {\scshape\Large \textbf{FACULTAD DE INGENIERÍA}\par}
        \vspace{0.2cm} % Ajuste mínimo para un espaciado más consistente
        {\Large Escuela Académico Profesional de Ingeniería Ambiental\par}
        \vspace{0.5cm} % Espacio ajustado entre "Ingeniería Ambiental" y "Tesis"
        {\LARGE\bfseries Tesis\par}
        \vspace{0.4cm} % Ajuste de espacio entre "Tesis" y el título de la tesis
        {\large\bfseries Estimación de Parámetros Fisicoquímicos para Evaluar la Calidad de Agua en la Laguna de Pacucha mediante Geoestadística, 2023\par}
        \vspace{1.2cm} % Ajuste de espacio entre el título de la tesis y el autor
        {\Large\bfseries Autor\par}
        \vspace{-5pt} % Reducir el espacio entre "Autor" y el nombre
        {\Large Virgilio Arriaga Gomez\par}
        \vspace{10pt} % Espacio ajustado antes del texto "Para optar..."
        {\Large Para optar el Título Profesional de \\[-3pt]Ingeniero Ambiental\par}
        \vspace{15pt} % Espacio ajustado antes del lugar y año
        {\Large Huancayo - Perú \\[-3pt]2023\par}
    \end{center}
    
    % Empujar las líneas gruesas completamente hacia la parte inferior
    \vfill % Empuja el contenido hacia arriba dejando espacio para las líneas al final de la página
    
    % Pegamos las líneas completamente al borde inferior
    \vspace*{0mm} % Ajustar el espacio para estar justo al final
    \noindent % Sin sangría en las líneas
    \makebox[\linewidth]{\rule{\paperwidth}{3mm}}\\[1mm] % Línea gruesa ocupando todo el ancho de la hoja
    \makebox[\linewidth]{\rule{\paperwidth}{3mm}}\\[1mm] % Línea gruesa ocupando todo el ancho de la hoja
    \makebox[\linewidth]{\rule{\paperwidth}{3mm}} % Línea gruesa ocupando todo el ancho de la hoja

\end{titlepage}

